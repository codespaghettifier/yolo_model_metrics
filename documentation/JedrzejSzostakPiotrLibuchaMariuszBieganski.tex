%%%%%%%%%%%%%%%%%%%%%%%%%%%%%%%%%%%%%%%%%
% Short Sectioned Assignment
% LaTeX Template
% Version 1.0 (5/5/12)
%
% This template has been downloaded from:
% http://www.LaTeXTemplates.com
%
% Original author:
% Frits Wenneker (http://www.howtotex.com)
%
% License:
% CC BY-NC-SA 3.0 (http://creativecommons.org/licenses/by-nc-sa/3.0/)
%
%%%%%%%%%%%%%%%%%%%%%%%%%%%%%%%%%%%%%%%%%

%----------------------------------------------------------------------------------------
%	PACKAGES AND OTHER DOCUMENT CONFIGURATIONS
%----------------------------------------------------------------------------------------

\documentclass[paper=a4, fontsize=11pt]{scrartcl} % A4 paper and 11pt font size

\usepackage[T1]{fontenc} % Use 8-bit encoding that has 256 glyphs
%\usepackage{fourier} % Use the Adobe Utopia font for the document - comment this line to return to the LaTeX default
\usepackage[english]{babel} % English language/hyphenation
\usepackage{amsmath,amsfonts,amsthm} % Math packages

\usepackage{lipsum} % Used for inserting dummy 'Lorem ipsum' text into the template

\usepackage{sectsty} % Allows customizing section commands
\allsectionsfont{\centering \normalfont\scshape} % Make all sections centered, the default font and small caps

\usepackage{fancyhdr} % Custom headers and footers
\pagestyle{fancyplain} % Makes all pages in the document conform to the custom headers and footers
\fancyhead{} % No page header - if you want one, create it in the same way as the footers below
\fancyfoot[L]{} % Empty left footer
\fancyfoot[C]{} % Empty center footer
\fancyfoot[R]{\thepage} % Page numbering for right footer
\renewcommand{\headrulewidth}{0pt} % Remove header underlines
\renewcommand{\footrulewidth}{0pt} % Remove footer underlines
\setlength{\headheight}{13.6pt} % Customize the height of the header

\numberwithin{equation}{section} % Number equations within sections (i.e. 1.1, 1.2, 2.1, 2.2 instead of 1, 2, 3, 4)
\numberwithin{figure}{section} % Number figures within sections (i.e. 1.1, 1.2, 2.1, 2.2 instead of 1, 2, 3, 4)
\numberwithin{table}{section} % Number tables within sections (i.e. 1.1, 1.2, 2.1, 2.2 instead of 1, 2, 3, 4)

\setlength\parindent{0pt} % Removes all indentation from paragraphs - comment this line for an assignment with lots of text

\usepackage{datetime}
\usepackage{polski}
\usepackage[utf8]{inputenc}

\usepackage{amsfonts}
\usepackage{amsmath}
\usepackage{relsize}
\usepackage{mathrsfs}

\usepackage{graphicx}
\usepackage{xcolor}
\usepackage{ragged2e}

%----------------------------------------------------------------------------------------
%	TITLE SECTION
%----------------------------------------------------------------------------------------

\newcommand{\horrule}[1]{\rule{\linewidth}{#1}} % Create horizontal rule command with 1 argument of height

\title{	
\normalfont \normalsize 
\textsc{Wydział Fizyki i Informatyki Stosowanej AGH} \\ [25pt] % Your university, school and/or department name(s)
\horrule{0.5pt} \\[0.4cm] % Thin top horizontal rule
\LARGE Uczenie maszynowe \\ projekt grupowy
\\ % The assignment title
\vspace{5mm}
\textsc{Zgłoszenie tematu}
\horrule{2pt} \\[0.5cm] % Thick bottom horizontal rule
\textsc{Określenie niepewności przewidywań modelu YOLO na ogólnie dostępnych benchmarkach.}
}

\author{	Autorzy: \\
		\textbf{Jędrzej Szostak}, 
		\textbf{Piotr Libucha}, 
		\textbf{Mariusz Biegański}} % Your name

\date{\normalsize\today} % Today's date or a custom date

\begin{document}

\maketitle % Print the title

%----------------------------------------------------------------------------------------
%	PROBLEM 1
%----------------------------------------------------------------------------------------

\section{Cel i zakres projektu}
    Celem projektu jest określenie niepewności przewidywań modelu YOLO w zakresie liczby i poprawności wykrytych obiektów.
\section{Oprogramowanie}

	YOLO może działać w oparciu o bibliotekę Darknet albo o Tensorflow

 https://pjreddie.com/darknet/yolo/

\section{Dane wejściowe}

	Należy zgromadzić jak największy zbiót otagowanych danych. Jako pukt wyjścia zostaną wzięte standardowe benchmarki używane dla YOLO.
\section{Kamienie milowe projektu}
\subsection{1-sze 2 tygodnie}
    Implementacja i testy eksploracyjne bazowej wersji modelu (Jędrzej Szostak 33\% Piotr Libucha 33\% Mariusz Biegański 33\%)
\subsection{2-gie 2 tygodnie}
    Reprodukcja dotychczasowych benchmarków jako test architektury.
    (Jędrzej Szostak 33\% Piotr Libucha 33\% Mariusz Biegański 33\%)
\subsection{3-cie 2 tygodnie}
    Przygotowanie docelowego środowiska roboczego, tj. modelu i danych oraz wybór metody określania niepewności (jackknife, bootstrap, dropout)
    (Jędrzej Szostak 33\% Piotr Libucha 33\% Mariusz Biegański 33\%)
\subsection{4-te 2 tygodnie}
    Eksperymenty numeryczne (Jędrzej Szostak 33\% Piotr Libucha 33\% Mariusz Biegański 33\%)
\subsection{5-te 2 tygodnie}
    Opracowanie wyników (Jędrzej Szostak 33\% Piotr Libucha 33\% Mariusz Biegański 33\%)
\section{Analiza parametryczna}

	Szczegółowe omówienie poszczególnych etapów i operacji składowych zaplanowanych na potrzeby poszukiwania optymalnych parametrów klasyfikatora. W tej części można zamieszczać fragmenty kodu źródłowego, jeśli zachodzi taka potrzeba oraz schematy blokowe.
	
\section{Wyniki eksperymentalne}

	Rezultaty oceny poprawności działania programu w formie tabel/wykresów.

\section{Wnioski}

	Analiza uzyskanych w części eksperymentalnej rezultatów, przedstawienie wniosków i~obserwacji.
	
\end{document}
